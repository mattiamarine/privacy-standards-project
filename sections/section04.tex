\section{ISO/IEC 29134}
\label{sec:iso-iec-29134}

\begin{figure}[H]
    \centering
    \includegraphics[width=0.5\textwidth]{images/iso_iec_29134.png}
    \caption{ISO/IEC 29134 Guidelines for privacy impact assessment}
    \label{fig:iso-iec-29134}
\end{figure}

The \textbf{ISO/IEC 29134 standard} provides guidelines for conducting \textbf{Privacy Impact Assessment (PIA)} 
to help organizations identify and mitigate privacy risks associated with their projects, systems or processes,
to all types of initiatives that involve the processing of personally identifiable information (PII).

A Privacy Impact Assessment \cite{iso_iec_29134_2020} is a process that helps assessing the \textbf{potential
impact} on privacy of a process, system, programme, software application, device or other initiative
that processes personally identifiable information (PII). A PIA leads, 
along with stakeholder consultations, to the \textbf{identification of the best
actions} to mitigate privacy risks. A report is produced at the end of the PIA process
and it may include documentation about the measures taken to address the risks.

\vspace{\baselineskip}
A PII controller may have the responsibility to conduct a PIA and may 
request to a PII processor to assist in the PIA process, acting on the 
PII controller's behalf. A PII processor or a supplier may also conduct 
a PIA on its own initiative. 

\vspace{\baselineskip}
A PIA is typically conducted by an organization that takes its responsibility seriously and treats PII 
principals adequately. In some jurisdictions, a PIA may be necessary to meet legal and regulatory 
requirements.

\vspace{\baselineskip}
Controls deemed necessary to treat the risks identified during the privacy impact analysis process
may be derived from multiple sets of controls, including ISO/IEC 27002 (for security controls) and 
ISO/IEC 29151 (for PII protection controls).

\subsection{Scope}
As just mentioned, the ISO/IEC 29134 standard provides guidelines 
for conducting Privacy Impact Assessments (PIAs) and for the structure
and content of PIA reports. With this purposes, the standard is applicable
to \textbf{all types of organizations, regardless of their size}, including
public and private companies, government entities and non-profit organizations.
The standard is also applicable to all types of initiatives that are involved
in designing or implementing projects or systems that process personally identifiable information (PII).

\subsection{Relationship with Other Standards}
ISO/IEC 29134 does not operate in isolation but \textbf{is part of a broader ecosystem of standards}
related to information security and privacy management. While ISO/IEC 29100 described in Section~\ref{sec:iso-iec-29100} provides
a high-level framework for privacy principles, defining terminology, actors, roles,
privacy principles and privacy requirements, focusing on the \textbf{"what"} of privacy management,
ISO/IEC 29134 focuses on the \textbf{"how"}, providing a methodological implementation of the privacy principles
defined in ISO/IEC 29100 through the PIA process.

Furthermore, ISO/IEC 29151 described in Section~\ref{sec:iso-iec-29151} defines
specific controls for the protection of PII, such as access control policies,
logging, monitoring, encryption, key management, and many others, focusing on \textbf{"how to protect PII"}.
ISO/IEC 29134 interacts with ISO/IEC 29151 by identifying the privacy risks
that need to be mitigated through the implementation of the controls defined 
in ISO/IEC 29151, and the PIA report may document the implementation of such controls.
For instance, if a PIA (ISO/IEC 29134) identifies a risk related to unauthorized access to PII,
ISO/IEC 29151 provides the necessary controls to mitigate that risk, which
include control for cryptographic protection, for secure channels and for authentication
mechansims.

\subsection{Key Concepts and Definitions}
The terms and definitions used in ISO/IEC 29100 (see Section~\ref{sec:iso-iec-29100}) and ISO/IEC 27000
still apply in ISO/IEC 29134. In addition to those, ISO/IEC 29134 introduces specific terms related to Privacy Impact Assessments (PIAs).

\subsubsection{Asset}
An \textbf{asset} is any data, device or other component that \textbf{has a value} to anyone involved in the processing of PII. 
In the context of a PIA process, an asset is either PII or a supporting asset.

\subsubsection{Assessor}
The \textbf{assessor} is the person or group of people responsible for conducting the PIA.
The assessor may be an \textbf{internal resource} of the organization or an \textbf{external consultant} and
may be supported by one or more other internal and/or external experts.

\subsubsection{Privacy Impact}
A \textbf{privacy impact} is anything that has an effect on the privacy of PII principal or a group of PII principals.
The privacy impact could be the \textbf{result of the processing of PII} in a way that is not compliant
with applicable privacy requirements.

\subsubsection{Privacy Impact Assessment (PIA)} 
A \textbf{Privacy Impact Assessment (PIA)} is a process of identifying,
analysing, evaluating, communicating and planning the treatment 
of privacy risks related to the processing of PII.

\subsection{Preparing the grounds for PIA}
The standard provides guidance that can be adapted to a wide range of situations where PII is 
processed. However, in general, a PIA can be carried out for the purpose of \textbf{identifying privacy impacts, privacy risks and responsibilities}, 
providing input to design for privacy protection, reviewing a new information system's privacy risks and assessing its impact and likelihood,
sharing and mitigating privacy risks with stakeholders, or providing evidence relating to compliance.
It provides a way to detect potential privacy risks arising from the processing of PII and thereby informing an organization of where they 
should take precautions and build tailored safeguards before, not after, the organization makes heavy 
investments. Amending a project at the planning stage is generally much less costly than making changes after implementation.

\vspace{\baselineskip}
A PIA can contribute to an organization's demonstration of its \textbf{compliance} with relevant privacy 
and data protection requirements. In the event of privacy risk or breach occurring, 
the PIA report can provide \textbf{evidence} that the organization acted 
appropriately in attempting to prevent the occurrence. This can help to reduce or even eliminate any 
liability, negative publicity and loss of reputation. It can also help to 
build \textbf{trust and confidence} among customers, as it also demonstrates that it respects
their privacy and is committed to protecting their personal data.

\vspace{\baselineskip}
PIA reporting should fulfil two basic functions. The first \textbf{keeps the specific stakeholders 
informed} of identified affected entities, affected environment and privacy risks about the life cycle 
of the affected entities. The second is a \textbf{tracking 
mechanism} on the actions/tasks that improve and/or resolve the identified privacy risks.

\subsection{Conducting a PIA}

\subsubsection{Determine if a PIA is necessary}
In order to determine whether a new or updated PIA is necessary, the organization should consider
information about the programme, information system or process that is being assessed,
with the purpose of performing a \textbf{threshold analysis} and to prepare a new or updated PIA if required.

\vspace{\baselineskip}
If a \textbf{new or updated PIA} is required, the organization, in collaboration with the assessor, should 
define the terms of reference and determine the boundaries and applicability of the PIA in order 
to establish its scope. The organization should decide on the scale of the PIA, the process
to be used and the target audiences.

A new or updated PIA may be necessary if the organization perceives impacts on privacy
from a new technology, service or other initiative where PII is processed, or from a decision 
that sensitive PII will be processed, from changes in applicable privacy
related laws, regulations, standards or from business expansion or acquisition.

\subsubsection{Preparation of the PIA}

\paragraph{Set up the PIA team}
The organization should determine which is the \textbf{scope} of the PIA
and identify the \textbf{needed expertise} to conduct the PIA.
A person responsible for coordinating the PIA, that is the assessor, 
should be identified and appointed, as well as the person accountable
for signing off the PIA report. 

The assessor, on his side, should define the \textbf{risk criteria} and the criteria for 
\textbf{risk acceptance}, and ensure that senior management agrees with the risk 
criteria to be used to evaluate the significance of risk. The criteria should reflect 
the organization's values, objectives and resources, and the assessor
should consider legal, regulatory and contractual requirements, external 
factors such as industry guidelines, professional standards and other factors that can 
affect the design of information systems and associated privacy requirements.
These criteria should be used later for privacy risk assessment and treatment.

The output of this process should be included in the PIA report.

\paragraph{Prepare a PIA plan}
A \textbf{plan for conducting the PIA} should be prepared and human resources
as well as budget should be allocated. The plan should include
the \textbf{steps to be performed} and the \textbf{actions to be taken}, in terms of tasks and 
needed time and resources. The plan should also include the identification
of the \textbf{costs and level of effort} needed to conduct the PIA, deciding on the 
allocation of budget and resources. 

\vspace{\baselineskip}
The plan should take into account the scope of the assessment being undertaken, allowing for iterations 
if necessary. This should include iterations of the PIA report. This is particularly useful when the 
assessment involves large-scale resources, but may not be necessary for smaller, less complex initiatives. 
The plan should spell out \textbf{what} is to be done to complete the PIA, \textbf{who} on the PIA team will do what, 
the \textbf{PIA schedule} and especially how the consultations, if any, will be carried out. It should specify 
why it is important to consult stakeholders in this specific instance, who will be consulted, and how 
they will be consulted 

\vspace{\baselineskip}
Once the assessor has prepared the PIA plan, they should estimate the \textbf{costs} of undertaking the PIA 
and seek the \textbf{budgetary and human resources} necessary from the organization's senior management. 
The plan may require an increase in the nominal budget initially set by the senior management, or the 
person responsible for conducting a PIA may need to revise the PIA plan based on the budget available.

\paragraph{Describing what is being Assessed}
The organization should provide all the necessary information
about the programme, system requirement information, system design information, operational plans,
with the goal of providing a clear description of the business process and information system to be assessed.

By establishing the \textbf{context}, the organization defines the relevant \textbf{internal and external parameters} to 
be taken into account when managing privacy risk and setting the scope and privacy risk criteria for 
the remaining process. In order to gain a clear view of the scope under consideration, the assessor should seek answers to
questions such as what are the PII that will be processed, what is the purpose of the processing,
what are the main benefits offered by the processing, who are the PII recipients, which
principals are affected by the processing, and many other non-exhaustive questions.

\vspace{\baselineskip}
For each PII, the organization should identify the \textbf{supporting assets} (on which the PII rely) that will be 
used or that are being used. It should identify the \textbf{location of those supporting assets}. 
For the information system and supporting assets identified, the assessor should consult the commonly 
used operational plans and procedures with their underlying concepts. 

\paragraph{Identify the stakeholders and establish a consultation plan}
The organization should identify all the \textbf{stakeholders} (including PII principals) who might process PII or 
who might be impacted by the processing of their PII.  Examples of stakeholders may include employees, PII principals,
worker and consumer representatives, business partners, computer or network administrators,
application operators, maintenance people, and others.

\vspace{\baselineskip}
In order to make the PIA process \textbf{transparent} and achieve goals of the PIA for addressing the privacy 
risks, the person responsible for conducting a PIA should identify in detail the internal or external 
stakeholders who may have an interest in or be affected by either the process subject to a PIA or in the 
protection of their PII under this process. 

\vspace{\baselineskip}
The \textbf{scope and scale} of the PII will be important in determining the appropriate stakeholders. Where a 
large government project is being undertaken, there may be many stakeholders. In this case, societal 
interest groups such as consumer representatives may need to be identified as well as stakeholders 
who process PII and who are PII principals. In contrast, smaller commercial processing operations may 
not need to identify such a wide stakeholder list.

\vspace{\baselineskip}
Moreover the assessor should establish a \textbf{consultation plan}, defining the methods and schedule for consulting
the identified stakeholders during the PIA process. This plan should address 
issues relating to the impact on the various privacy stakeholders, their 
consequences (if known) and the measures being taken to manage them.  
The range and number of stakeholders to be consulted should be a function 
of the \textbf{privacy risks} and the assumptions about the \textbf{frequency 
and level of impact} of those risks and the \textbf{numbers of citizen
consumers} who could be impacted. For example, if the risks are likely 
to impact only the employees of a single organization, then the 
consultation could be limited to employees and/or their representatives. 
If, however, the risks are expected to impact everyone in the country, 
then the organization should consult widely with external stakeholders. 

\vspace{\baselineskip}
Feedback from stakeholders may identify issues that are \textbf{perceptions of risk} 
rather than actual risks. These should not be discarded but
treated under wider stakeholder management issues to aid 
communications activity.

\subsubsection{Perform the PIA}\label{sec:perform-PIA}

\paragraph{Identify information flows of PII}
The person responsible for conducting a PIA should consult with others in the organization and perhaps 
external to the organization to describe the \textbf{PII flows} and specifically how PII is collected, who 
is accountable and who is responsible within the organization for the PII processing, for what
purpose PII is processed, how PII is processed, PII retention and disposal privacy and many other relevant aspects.

\vspace{\baselineskip}
As an input to the PIA, the organization should describe the information flow 
in \textbf{as detailed a manner as possible} to help identify potential privacy risks. 
The assessor should consider the impacts not only on information privacy, 
but also compliance with privacy related regulations, e.g., telecommunications 
acts. The whole PII life cycle should be considered.

\paragraph{Determine the relevant privacy safeguarding requirements}
The organization should identify the relevant \textbf{privacy safeguarding requirements}
and the person responsible for conducting a PIA or their legal experts should ensure the business process 
under scope complies with any legislative, regulatory, business factors and contractual requirements 
regarding privacy and/or data protection.

\paragraph{Assess privacy risk}
Organizations should identify \textbf{privacy risks} to be assessed by applying 
\textbf{privacy risk identification tools} and techniques which are suited to 
its objectives and capabilities, and to the risks faced. The respective legal privacy principles of the 
country in which the solution will be deployed should be used to support the identification of risks for 
privacy breach. 

\vspace{\baselineskip}
Privacy risks include, but are not limited to unauthorized access to PII,
unauthorized modification of PII, unauthorized removal or theft of the PII,
excessive collection or inappropriate linking of PII, processing of PII without 
the knowledge or consent of the PII principal,
unnecessarily prolonged retention of PII, and many others. 
Relevant and up-to-date information is important in identifying privacy risks. 
The assessor should involve people with appropriate knowledge in identifying privacy risks. 
After identifying what might happen, it is necessary to consider 
possible scenarios that show what consequences can occur. 

\vspace{\baselineskip}
\textbf{Risk analysis} includes identifying the PII and supporting assets that may be at risk, 
the vulnerabilities associated with those assets, the threats that might exploit those vulnerabilities, the
likelihood and impact of that happening, as well as any existing controls that might influence the risk.
Privacy risk analysis involves consideration of the causes and sources of privacy risk, their positive and 
negative consequences, and the likelihood those consequences can occur. The assessor should identify 
factors that affect consequences and likelihood. An event can have multiple consequences and can affect 
multiple objectives. The assessor should take into account existing controls and their effectiveness.

\vspace{\baselineskip}
If a privacy risk is assessed as having a \textbf{high or very high impact} and/or is likely or very likely to occur, 
the organization should consider decomposing that risk into its \textbf{sub-elements}. This decomposition will 
allow the organization to identify which sub-element is contributing to the high impact or likelihood 
and analyse it separately. This should help identify more appropriate controls.

\vspace{\baselineskip}
To estimate the likelihood of the threats, the organization should 
consider the risk sources capacities, the vulnerabilities of the supporting assets
and the existing or planned controls. To better understand \textbf{how to estimate the level of impact
and the likelihood}
of identified threats, refer to Section~\ref{sec:impact-and-likelihood}.

\vspace{\baselineskip}
Privacy risk analysis can be undertaken with varying \textbf{degrees of details} depending on the privacy risk, 
the purpose of the analysis and the information, data and resources available to support the analysis. 
Analysis can be \textbf{qualitative, semi-quantitative or quantitative}, or a combination of these, depending on 
the circumstances. In practice, qualitative analysis is often used first to obtain a general indication of the level of risk and 
to reveal the major risks. When possible and appropriate, one should also undertake more specific and 
quantitative analysis of the risks.

\paragraph{Treating privacy risks}
The \textbf{privacy risk treatment options} should be identified and the most appropriate options should be selected.
Risk treatment may include, but is not limited to, conducting application or process redesign, depending 
on the scope of the assessment, context of risk management, or industry sector.
Selecting the most appropriate privacy risk treatment option involves \textbf{balancing the costs and efforts} 
of implementation against the organization's obligation for protecting the privacy of any stakeholder 
whose privacy might be impacted by the organization.

\vspace{\baselineskip}
The privacy risk evaluation can also lead to a decision not to treat the privacy risk in any way other 
than maintaining existing privacy controls. This decision will be influenced by the organization's \textbf{risk 
appetite or risk attitude} and the privacy risk criteria that have been established.
Wherever appropriate, an organization and/or the assessor should seek stakeholders support in the 
selection of privacy risk treatment options. When selecting risk treatment options, the organization should consider the values and perceptions of 
stakeholders and the most appropriate ways to communicate with them. Where privacy risk treatment 
options can impact on risk elsewhere in the organization, these areas should be involved in the decision.

\vspace{\baselineskip}
There are four options available for privacy risk treatment: 
\begin{itemize}
    \item \textbf{Risk reduction}: can be achieved through the selection of appropriate controls. If after control 
    selection there is still some \textbf{residual risk}, the organization should determine whether the residual 
    risk is unacceptable and should be further addressed through the selection of additional controls, or 
    determined to be acceptable to the organization and to the stakeholders.
    Risk reduction controls will be varied in nature. They may involve changes to the kind of PII being processed,
    changes to organizational structure, policies and/or procedures or changes to personnel qualifications.

    \item \textbf{Risk retention}: if the level of risk meets the risk criteria, there is no need for implementing additional controls and the 
    risk can be retained. 

    \item \textbf{Risk avoidance}: when the identified risks are considered too high, the organization should decide to avoid the risk 
    completely, by withdrawing from a planned or existing activity or set of activities, or changing the 
    conditions under which the activity is operated.

    \item \textbf{Risk transfer}: involves a decision to \textbf{share} certain risks with external parties. Transfer can be done by 
    \textbf{insurance} that will support the consequences, or by \textbf{sub-contracting} a partner whose role will be to 
    monitor the information system and take immediate actions to stop an attack before it makes a defined 
    level of damage. Risk transfer can create new risks or modify existing, identified risks. Therefore, additional risk 
    treatment may be necessary.
    It may be possible to transfer the responsibility to manage risk but it is not normally possible to transfer 
    the liability of an impact. Stakeholders will usually attribute an adverse impact as being the fault of the 
    organization.
\end{itemize}

\vspace{\baselineskip}
The appropriate controls for the selected risk treatment options as well as the legally required controls 
should be identified. Controls that are already existing or planned that met the requirement of the ISO/IEC 29100 principles 
should be described and evaluated. \textbf{Additional controls} may be selected from existing control sets defined in recognized international 
standards or issued by recognized institutions. They may also be developed by the organization 
independently of any existing control sets. If necessary, controls should be adapted to the specific 
context of the programme, information system or process under consideration.

If a control (or controls) is not selected to meet a compliance requirement, the organization 
should document its rationale for not selecting a control (or controls). Reasons for not selecting a control 
could include the lack of a suitable control, or the controls that are available are prohibitively expensive 
in relation to the value of the asset to be protected.

\vspace{\baselineskip}
One or more privacy risk treatment plan(s) should be formulated and it should estimate the cost of implementing each control.
The information provided in treatment plans should include what privacy safeguarding requirements protect against which risks, 
a list of PII including nature and ownership of the PII to be protected, performance measures and constraints, proposed actions, 
reporting and monitoring requirements, timing and scheduling, and many other relevant aspects. 
The risk owner should approve the privacy risk treatment plan and accept the residual privacy risks. 

\subsubsection{Follow up the PIA}

\paragraph{Prepare the report}
The penultimate step of the PIA process is the \textbf{recording of the results}
of each of the previous steps in a comprehensive report. In this step,
it should be decided which elements of the PIA report should be published
and which elements should be shared with the stakeholders involved in the PIA process.
The assessor should \textbf{prepare the PIA report and sign off} in the following step.

\vspace{\baselineskip}
For this purpose, the assessor should collect the output from the previous
steps in the report. This report should be filed by the person who is 
responsible for conducting the PIA and should be approved by the organization's
management responsible for the process, system or programme that 
is processing PII.

Once ready, the report should be sent to the memeber of the organization
that has commisioned the PIA for review and approval. 

\paragraph{Publication}
The last step of the PIA process is the \textbf{publication} of the PIA report 
and a \textbf{PIA public summary}, if applicable.

The organization should mantain a registry of PIA reports, that could be
a simple listing of the PIAs, their titles and the dates the PIA reports
were published. This registry should be made available to visitors, 
especially PII principals to its website and they should be able to download
a copy of any published PIA material.

\paragraph{Implement privacy risk treatement plans}
Once the PIA report has been signed off and published, several actions need 
to be performed to \textbf{implement the privacy risk treatment plans} defined
during the PIA process. 

\vspace{\baselineskip}
Before implementing the processing of PII, where possible, the organization should:
\begin{itemize}
    \item Provide adequate \textbf{training} to personnel involved in the project 
    to make sure they are sensitive to the privacy implications, the possible 
    impacts on privacy and what they or their collegues should do.

    \item Provide the \textbf{budget} for implementing the selected controls
    
    \item Make both the \textbf{privacy policy} and \textbf{privacy practice statement} availabe to end-users, 
    in case the service that is the object of the project is made available to the end-users
    
    \item Implement the \textbf{treatment plan}
\end{itemize}

\vspace{\baselineskip}
The risk owner and the organization may not accept all of the PIA report's recommendations.
In this case, the organization should document the \textbf{reasons for not accepting}
the recommendations and communicate them to the assessor. Moreover, 
the organization should monitor the implementation of the
privacy risk treatment plans and the effectiveness of the implemented controls.

\paragraph{Review and/or audit of the PIA}
An organization may decide to review and/or audit the PIA process, establishing
a policy for reviews and/or audits setting out types, schedules and thresholds that
trigger them.

Independent review or audit of the PIA is a way of ensuring the PIA has been conducted appropriately 
and the organization has implemented the risk treatment plan or, if it has not implemented some 
recommendations, then it can say why it has not done so.

\vspace{\baselineskip}
Third-party review or audit, where possible, is a way of giving \textbf{credibility} to the PIA report, of 
improving \textbf{transparency}, of learning from experience and raising the quality of PIA practices. If the PIA 
is conducted by a third party, then the review and/or audit of the PIA should not be carried out by the 
same third party.

\paragraph{Reflect changes to the process}
If there are significant changes in the business process affecting 
the processing of PII, the organization should have a mechanism for \textbf{updating
the PIA report}. The organization should explain why changes are being made 
and how these changes might affect the processing of PII. It should 
verify that the processing of PII meets privacy safeguarding requirements
by periodically conducting an internal audit or a trusted third-party audit.

If the business process changes, or even after a previously defined
period of time, a \textbf{new PIA} should be conducted. Otherwise, an organization
can decide if it is preferable to update the PIA with a simple audit, for instance, 
in order to check whether basi conditions have changed.

\subsection{PIA Report}
This section provides guidance on the content of the \textbf{PIA report}.
The PIA report's contents depend strongly on the type and sensitivity of 
the PII being processed, on its nature and on the scope of the PIA being conducted.

Some of the PIA report's details may be confidential. They may address business issues that should not 
be made public. They may address treatment options that may reveal sufficient details about residual 
risks to increase the risk of system compromise. The organization should determine the appropriate \textbf{audience} for and contents of the PIA report and 
its degree of confidentiality. A report provided in confidence to an independent auditor or to a data 
protection authority may contain more details than one provided to stakeholders or to the public.

\subsubsection{Report Structure}
The PIA report should be adapted to specific circumstances determined during the 
earlier steps of the PIA process. However, in general, the PIA report should include
a \textbf{cover page} with at least the name of the process, system or programme being assessed,
the name and address of the PII controller and the organization
that is conducting the PIA, the contact person along with contact details, the 
date of the report and should also contain the name of those to whom 
one can address questions about the report.

\vspace{\baselineskip}
It must include an \textbf{introduction} that explains why the PIA has been conducted, when it has
been conducted and who was involved in the PIA process. It should contain
\textbf{information about the process}, system or programme being assessed, including
its purpose, scope and objectives, as well as any other contextual information
about the organization and its environment that can be relevant 
for understanding the rationale behind the PIA.

The \textbf{main body} of the report should contain the details of each step
of the PIA process. If the PIA report is long, it should also 
include an \textbf{executive summary} that provides the main findings and reccomandations
of the PIA and which stakeholders were consulted. It should 
indicate why the PIA was conducted, who initiated the PIA and who conducted it.
It should also identify the principle privacy impacts and the alternatives for minimizing or 
avoiding negative impacts.

\subsubsection{Scope of the PIA}
The PIA report should clearly define the \textbf{scope} of the PIA. The organization
should provide the \textbf{most complete description possible} of the process, system or programme
being assessed in order to define the boundaries of the PIA. 

The PIA report should state how individuals are \textbf{notified} that the organization is collecting information 
about them, and what role individual consent plays in the process, information system or programme. 
The organization should say how it intends to \textbf{delete} the PII once it is no longer needed. It should say 
what procedures it will put in place to allow individuals to see their PII and to rectify it if necessary or 
to request its deletion. The organization should also specify the \textbf{costs}, if any, of allowing 
individual access to their PII and how much time it takes the organization to respond to requests.

\vspace{\baselineskip}
The PIA report should describe at least \textbf{system requirements}, the \textbf{system design} and the \textbf{operational 
plans and procedures} as described in the following paragraphs.

\paragraph{System Requirements}
The PIA report should include a description of the system requirements, including:
\begin{itemize}
    \item the \textbf{purpose} of the processing of PII
    \item a description of the \textbf{business process} that is supported by the information system
    \item a list of all the \textbf{functional requirements} defined for the information system
    \item the \textbf{information security requirements}
    \item a description of \textbf{how data will be collected, from whom and why}
    \item a statement on the \textbf{justification for processing the PII involved}
\end{itemize}

\paragraph{System Design}
The PIA report should include a description of the system design, including:
\begin{itemize}
    \item a description of the \textbf{functional and physical architecture} of the information system
    \item the s\textbf{tructure and list} of information system components (databases, tables, fields that contain PII)
    \item a data flow diagram showing the \textbf{life cycle of the PII}
    \item a work flow diagram that describes \textbf{when to notify and get consent} from PII principals
    \item a list of \textbf{interfaces}, defining the parties connected and the data fields transferred
    \item details of \textbf{ports, protocols, APIs and encryption detail}
\end{itemize}

\paragraph{Operational Plans and Procedures information}
The PIA report should include a description of the operational plans and procedures information, including:
\begin{itemize}
    \item the \textbf{identity and user management concept} for the information system
    \item the \textbf{operational concept} including if the information system is operated on 
    site or is externally hosted, or cloud sourced and where
    \item the \textbf{support concept}, especially listing third parties by name who are involved in supporting the 
    information system
    \item the \textbf{logging concept} and the respective retention plans
    \item the \textbf{backup and recovery plans}
    \item the \textbf{protection and management of metadata}
    \item the \textbf{data retention and deletion plans}
\end{itemize}

\vspace{\baselineskip}
The PIA report should also describe the \textbf{chosen risk criteria} and should at least contain
the criteria to estimate the level of impact and the likelihood of the identified privacy risks,
the scales used for the estimation, as well as the criteria for risk acceptance.

\vspace{\baselineskip}
The report should also provide a statement on the \textbf{composition of the PIA team}, 
the \textbf{major milestones} of the PIA process and the \textbf{budget and resources spent}. Moreover,
the organization is expected, during the PIA process, to have identified the types of stakeholders to be consulted; 
the report should include a list of those stakeholders and a summary of the consultation plan, including
the result of the consultations.

\subsubsection{Privacy Requirements}
The PIA report should list the \textbf{relevant sources for the requirements identified} by the PIA team as 
needing to be met. These may include legal, regulatory, business and contractual requirements. The report should
also include a description of how these requirements have been identified and analysed.

\subsubsection{Risk Assessment}
The PIA report should include a description of the \textbf{sources of privacy risks} identified, 
the determined \textbf{threats and vulnerabilities} that may allow identified risks to occur and 
the estimated \textbf{impact and likelihood} of the identified privacy risks.
The report should also provide a privacy risk map showing the level
of the impact and the likelihood of each identified privacy risk. 
Depending on where the risks are located on the map and depending on the risk
criteria, \textbf{priorities} should be set.

\subsubsection{Risk Treatement Plan}
The PIA report should record the \textbf{risk treatement plan} and the stage of implementation
of each of the defined actions and controls.

\subsubsection{Conclusion and decision}
The PIA report should also record the decisions that have been made
during the PIA process on the \textbf{acceptance of residual risks}, on 
implementation of the \textbf{reccomandations} and on non-publishing elements of the
PIA report. 

\subsubsection{PIA Public Summary}
The assessor may decide to prepare a \textbf{PIA public summary}, that is a summary
of the PIA report that can be published and made available to the public. In this
way users, whether they are external PII principals or employees, can be provided 
with privacy risk information, to support consent. If necessary, the summary should 
remove commercially sensitive information that might be present in 
the full PIA report and include only those key aspects relevant to PII principals.

\vspace{\baselineskip}
The PIA public summary should include for instance:
\begin{itemize}
    \item the \textbf{benefits} of the programme, system or process being assessed
    \item the \textbf{types} of PII being processed
    \item \textbf{legal jurisdictions} under which the PII will be processed
    \item a summary of the \textbf{compliance analysis}
    \item a summary of the identified \textbf{privacy risks}
    \item a summary of the selected \textbf{risk treatment options}
    \item \textbf{contact details} for further information
\end{itemize}

All the above information and all additional information in a transparent, clear and 
comprehensible manner.

\subsection{Scale criteria on the level of impact and on the likelihood}
\label{sec:impact-and-likelihood}
In order to better understand how to estimate the level of impact and the likelihood
of identified threats, as covered in Section~\ref{sec:perform-PIA} under the \textit{Assess privacy Risk} paragraph,
the standard provides guidance on defining scale criteria for both impact and likelihood.

\subsubsection{How to estimate the level of impact}
The level of impact of the identified consequences should be estimated, taking into account those 
consequences and the planned or implemented controls. The answer one should seek is
\textbf{how much damage can be caused by all the potential impacts}. By answering this question, one can
determine the level of impact, that can be:
\begin{itemize}
    \item \textbf{Negligible}: the impact does not affect or has a very few inconveniences upon
    PII principals.

    \item \textbf{Limited}: PII prinicpals may experience significant inconveniences, but they will be able to 
    overcome despite some difficulties (such as extra costs, denial of access to business services, fear, stress, etc.)

    \item \textbf{Significant}: PII principals will experience considerable difficulties that may take
    a long time to overcome (such as blaklisting by banks, property damage, loss of employment, etc.)

    \item \textbf{Maximum}: PII principals will experience severe or even irreversible consequences, which they
    may not even overcome (such as financial distress, including debt or inability to work, long-term psychological or physical ailments, death, etc.)

\end{itemize}

\vspace{\baselineskip}
The level that best matches the estimated impact should be selected.
This level of impact can be later modified by including additional factors,
such as the number of affected PII principals, the level of sensitivity of the processed PII,
and the scale of the processing operation, which can be considered as aggravating factors.
On the other hand, the level of impact can be reduced by considering
additional factors such as poorly identifying PII, very few or no interconnections or recipients, and so on.
An example for the level of impact is given in Table~\ref{tab:level_of_impact}.

\begin{table}[h!]
    \centering
    \caption{Examples for the level of impact, based on the nature of the PII}
    \vspace{0.5cm}
    \begin{tabular}{|p{12cm}|c|}
        \hline
        \textbf{Nature of the PII} & \textbf{Level of impact} \\
        \hline
        PII that is \textit{publicly accessible} (e.g. in telephone directories, 
        address books or selection lists) 
        & 1 \\
        \hline
        PII that \textit{requires a legitimate interest} for access 
        (e.g. restricted public files or the members of a distribution list) 
        & 2 \\
        \hline
        PII whose unauthorized disclosure can \textit{affect the reputation} of the PII principal 
        (e.g. information about income, social welfare benefits, property tax or penalties) 
        & 3 \\
        \hline
        PII whose unauthorized disclosure, modification, loss or destruction can 
        \textit{affect the existence or the health, freedom and life} of the PII principal 
        (e.g. information about commitment to an institution, a sentence, personnel reviews, 
        health data, unserviceable debts, or if the PII principal is at risk of becoming a victim 
        in a criminal case) 
        & 4 \\
        \hline
    \end{tabular}
    \label{tab:level_of_impact}
\end{table}


\subsubsection{How to estimate the likelihood}
The likelihood of each threat being exploited should be estimated, taking into account the vulnerabilities 
of the supporting assets and the capabilities of risk sources to exploit them.
The answer one should seek is \textbf{what degree can the properties of supporting assets be exploited in order to carry out a threat}.
The likelihood can be:
\begin{itemize}
    \item \textbf{Negligible}: the threat cannot be carrioud out by exploiting the properties
    of the supporting assets.

    \item \textbf{Limited}: the threat can be carried out with some difficulties for the selected risk resource
    
    \item \textbf{Significant}: the threat can be carried out with relative ease for the selected risk resource
    
    \item \textbf{Maximum}: the threat can be easily carried out by the selected risk resource
\end{itemize}

\vspace{\baselineskip}
The level that best matches the estimated likelihood should be selected.
This level of likelihood can be later modified by including additional factors, 
such as access to the Internet, exchanges of data with foreign sites, interconnections 
with other systems, and so on, which can be considered as aggravating factors.
On the other hand, the level of likelihood can be reduced by considering
additional factors such homogeneous systems, systems with no interconnections, and so on.