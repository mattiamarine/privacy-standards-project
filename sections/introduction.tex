The internet has revolutionized communication, commerce, and countless other aspects of modern life. However, the integration of technology into daily activities has also raised significant privacy concerns. As digital engagement grows, there is an increasing need to establish standards that protect user privacy and data security.

\textbf{Digital privacy standards} play a crucial role in building trust between individuals, organizations, and governments in the digital sphere. By providing guidelines for ethical data collection and use, standards guarantee fundamental privacy rights while enabling responsible digital innovation. \cite{IEEE_DigitalPrivacy_RoleOfStandardsInDigitalPrivacy_2025}

%section
\subsection{What is Data Privacy and why is it important?}
\textbf{Data privacy} refers to the proper handling of personal data to maintain an individual's privacy rights. It involves the collection, storage, use, and sharing of personal information in a secure and ethical manner, ensuring that sensitive data is protected from unauthorized access or misuse.

Personal data can include various types of information, such as names, addresses, phone numbers, email addresses, financial details, medical records, and online activities. Data privacy aims to safeguard these informations and give individuals control over how their personal data are used.

\vspace{\baselineskip}
It's important to distinguish \textbf{data privacy} from \textbf{data security}, although the two concepts are closely related. 

\textbf{Data security} refers to the measures taken to protect data from unauthorized access, theft, or corruption. It involves implementing technical controls, such as encryption, access controls, and firewalls, to secure data from cyber threats.

While data security focuses on protecting data from external threats, \textbf{data privacy} is about ensuring that personal data is collected, used, and shared in a responsible and ethical manner, respecting individuals' privacy rights and preferences. Data privacy encompasses legal and regulatory compliance, as well as ethical considerations surrounding the handling of personal information.

Furthermore, data privacy is essential for maintaining trust and reputation, both for individuals and businesses. In a world where data breaches and privacy violations are becoming increasingly common, consumers are more cautious about sharing their personal information. By demonstrating a commitment to data privacy, organizations can build trust with their customers, fostering long-lasting relationships and enhancing their reputation.
\cite{alation_dataprivacy_2025}

\begin{figure}[H] 
    \centering
    \includegraphics[width=0.8\textwidth]{images/importance_data_privacy.png} % percorso e scala
    \caption{Importance of Data Privacy}
    \label{fig:importance_data_privacy}
\end{figure}

%section
\subsection{General Data Privacy Principles}
Exists \textbf{general data privacy principles} that appear in most frameworks and standards that are analyzed later in this report \cite{ibm_data_privacy_2025}: 
\begin{itemize}
    \item \textbf{Access}: Users have a right to know what data a company holds, should be able to access their personal data on demand and to update or amend that data as needed;
    \item \textbf{Transparency}: Users have a right to know who has their data and what they do with it. At the point of data collection, organizations should clearly communicate what they are collecting and how they intend to use it.
    \item \textbf{Consent}: Organizations should get user consent for data storage, collection, sharing or processing whenever possible. If an organization keeps or uses personal data without the subject's consent, it should have a compelling reason to do so, such as a public interest use or a legal obligation.
    \item \textbf{Quality}: Organizations should strive to ensure the data they collect and keep is accurate. Inaccuracies can lead to privacy violations.
    \item \textbf{Collection, retention and use limitation}: An organization should have a definite purpose for any data it collects. It should communicate this purpose to users and only use the data for this purpose.
    \item \textbf{Privacy by design}: Privacy should be the default state of every system and process in the organization. Any products the organization designs or implements should treat user privacy as a core feature and key concern.
\end{itemize}

%section
\subsection{The Growing Importance of Privacy Standards}
\textbf{Data privacy standards} refer to established codes of practice that dictate how personal user data should be collected, processed, managed and shared. They outline rules and best practices for handling private consumer information in the digital ecosystem.

\vspace{\baselineskip}
With data being collected at an unprecedented rate, privacy has never been more important. Privacy standards are vital in protecting individuals’ personal data and ensuring businesses handle it responsibly.

These standards are so critical because in recent years, several high-profile \textbf{data breaches} have brought the issue of data privacy to the forefront. These breaches have exposed the sensitive personal information of millions of individuals, resulting in significant consequences for both individuals and businesses.
According to a 2021 report by RiskBased Security, there were over 22 billion records exposed in data breaches during the first half of the year. The healthcare industry, for example, experienced a 47\% increase in data breaches.

\vspace{\baselineskip}

As an example, it is possible to consider the \textbf{2017 Equifax breach}. It is one of the largest credit reporting agencies in the United States and it experienced a massive data breach that compromised the personal information of over 147 million Americans. The stolen data included names, Social Security numbers, birth dates, addresses, and, in some cases, driver's license numbers. This breach had far-reaching implications, as the stolen information could be used for identity theft, financial fraud, and other criminal activities.

\vspace{\baselineskip}
Another rappresentative case is the \textbf{Facebook-Cambridge Analytica scandal}. In 2018, it was revealed that the political consulting firm Cambridge Analytica had improperly accessed the personal data of up to 87 million Facebook users. This data was then used to create targeted political advertising campaigns during the 2016 U.S. presidential election. The scandal raised concerns about the privacy practices of social media platforms and the potential misuse of personal data for political purposes.
\cite{alation_dataprivacy_2025}
\vspace{\baselineskip}

These are just two examples that highlight the importance of data protections standards and regulations. As a consequence, many countries decided to implement their own regulations to protect user data, remaining compliant with international standards.


%section
\subsection{Privacy Compliance}
One key component of data privacy is \textbf{compliance} with privacy laws. Businesses invest significantly in people, processes and technology to ensure compliance with applicable digital privacy standards.

In order to be compliant, a company needs to \cite{iubenda_dataprivacy_important_2025}:
\begin{itemize}
    \item Have \textbf{clear legal documents} that explain how the company collects and processes the data and how individuals can exercise their rights;
    \item Obtain \textbf{explicit consent} from users before collecting their data or tracking their online behavior;
    \item Have \textbf{strong security measures} in place to prevent unauthorized access or data breaches.
\end{itemize}

The consequences of non-compliance can vary, but they often include damages to your reputation, reprimands, liability damages, and hefty fines. The GDPR, for example, is known for its huge fines, which can reach up to €20 million or 4\% of the annual worldwide turnover. In addition, under these laws, individuals can often sue companies and seek compensation for damages resulting from non-compliance.


%section
\subsection{Overview of Privacy Standards: NIST and ISO/IEC standards}

In today's digital landscape, the \textbf{National Institute of Standards and Technology (NIST)} and the \textbf{International Organization for Standardization (ISO)} dominate the standards scene, 
offering more than just security guidelines: they provide a practical bridge between privacy requirements and technical implementation. 

\vspace{\baselineskip}
These frameworks serve three crucial functions \cite{pembrokeprivacy_nist_iso_2024}: 
\begin{itemize}
    \item they transform high-level privacy principles into specific, actionable tasks; 
    \item facilitate seamless collaboration between privacy and cybersecurity teams through a common language and methodology;
    \item provide a globally recognized approach that extends beyond regional regulations like the GDPR, ensuring privacy and security practices remain relevant across international boundaries.
\end{itemize}


In particular \textbf{NIST SP}, also known as the National Institute of Standards and Technology \textbf{Special Publication}, is a series of publications developed by the NIST in the United States that provide detailed guidance and recommendations on a wide range of topics related to cybersecurity and information security.

In this report, the focus is on \textbf{NIST SP 800 series}, developed to address and support the security and privacy needs of U.S. Federal Government information and information systems.

\vspace{\baselineskip}
On the other hand, the family of standards analyzed is the \textbf{ISO/IEC 29000 series}, designed for global applicability, independent of legal jurisdiction. 
While NIST gives practical and technical guidance, ISO 29xx standards focus more on normative specifications, in which roles and concepts are standardized.

\vspace{\baselineskip}
In this report some specific standards from both frameworks will be analyzed, and at the end will be easier to understand how they complement each other to provide a comprehensive approach to data privacy and security.

%section
\subsection{Privacy Standards Analysed}
The privacy standards that will be analyzed in this report are the following:
\begin{itemize}
    \item \textbf{ISO 29100}: Privacy framework;
    \item \textbf{NIST SP 800-122}: Guide to Protecting the Confidentiality of Personally Identifiable Information (PII);
    \item \textbf{ISO 29151}: Code of practice for personally identifiable information protection;
    \item \textbf{ISO/IEC 29134}: Privacy impact assessment.
\end{itemize}

