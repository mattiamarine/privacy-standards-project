\section{ISO/IEC 29100}

\subsection{Overview on the standards and its objectives}
ISO/IEC 29100 provides a high-level privacy framework for the protection of personally identifiable information (PII) 
within information and communication technology (ICT) systems. The demand for stronger privacy protection
has increased significantly in recent years, especially with the rise of AI, IoT and global data sharing. 
The standards aims to provide the foundation for building privacy-by-sdesign systems, ensuring that privacy considerations
are aligned with technological, as well as regulatory, requirements.

For this purpose the privacy framework:
\begin{itemize}
    \item specifies a common privacy terminology,
    \item defines the actors and their roles in relation to PII,
    \item describes privacy requirements and
    \item references a set of known privacy principles.
\end{itemize}

The standard has been revised in 2024 to address modern challenges in cloud computing, data sovereignty and AI-driven data processing.

\subsubsection{Purpose}
The purposes of this standard are first of all to provide a common privacy terminology and structure that can be used
for implementing privacy controls and protection mechanisms across different ICT systems and services.
Privacy principles and the governance controls are defined in ISO/IEC 29100 and are intended to be applied to organizations, developers, 
service providers and regulators. The standard also align with international legal frameworks, such as the GDPR, the California Consumer Privacy Act (CCPA) and others,
in order to enable global privacy assurance and compliance. 

\subsubsection{Scope and Applicability}
As stated in the official documentation, ``ISO/IEC 29100 is applicable to natural persons and organizations involved in specifying, procuring, architecting, designing,
developing, testing, maintaining, administering, and operating information and 
communication technology systems or services where privacy controls are required for the processing of PII.''

This means that it is applicable to any organization, system or technology, regardless of the size or sector of the entity, which is involved in the processing of PII.
It is technology-neutral, that means that it can be applied to a wide range of ICT systems across on-premise sustems, cloud environments, mobile applications and 
emerging technologies such as AI or blockchain.

Applicability of the standard includes:
\begin{itemize}
    \item Data controlers and data processors,
    \item Software and system developers,
    \item Cloud service providers and digital platforms,
    \item Regulatory bodies and auditors,
    \item IT governance and privacy officers and
    \item Organizations seeking to enhance their privacy posture.
\end{itemize}

\subsection{Key Definitions}
\subsubsection{Personally Identifiable Information (PII)}
The first key definition provided by the standard is that of Personally Identifiable Information (PII).
PII is defined as information that can be used to establish a link between an information and a natural person to whom
this information relates, or an information that is or can be directly or indirectly linked to a natural person.

\subsubsection{PII controller}
The PII controller is the privacy stakeholder(s) that determines the purposes and means
for processing PII. Moreover, it can also be a natural person who use data for personal purposes.

Sometimes, a PII controller can devolve the responsibility of processing PII to another entity, such as a PII processor, 
who would process the data on behalf of the controller.

\subsubsection{PII processor}
The PII is the privacy stakeholder that processes PII on behalf of the PII controller and according to its instructions.

\subsubsection{PII principal}
The PII principal is the data subject, also referred to as the natural person, to whom the PII relates.

\subsubsection{Privacy control}
Another important definition is that of privacy control. A privacy control is a measure that 
can be implemented in order to reduce the likelihood of a privacy risk occurring or to mitigate the impact of a privacy risk.

Privacy controls include organizational measures, such as policies and procedures, as well as physical and technical measures, 
such as encryption, access controls and data minimization techniques. Control, in this context, is also used a synonym of safeguard or countermeasure.

\subsubsection{Privacy Enhanching Technology (PET)}
Privacy Enhanching Technology (PET) is any technology or privacy control consisting in ICT measures, product
or services that protect privacy by minimizing PII or by preventing unnecessary or unwanted processing of PII, 
without losing the functionality of the system.

Some examples of PETs include anonymization, encryption, pseudonymization tools that delete, reduce, mask or de-identify PII.
