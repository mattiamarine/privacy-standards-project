\section{Comparative Analysis}\label{sec:comparative-analysis}

The four standards analyzed in the previous sections address the protection of 
Personal Identifiable Infoormation (PII) from different perspectives and with varying levels of detail
but are yet \textbf{complementary} to each other. In particular, ISO/IEC 29100 provides a \textbf{high-level framework} for PII protection,
NIST SP 800-122 focuses on \textbf{practical guidelines} for safeguarding PII in IT systems,
ISO/IEC 29151 translates the high-level principles of ISO/IEC 29100 into specific \textbf{operational controls} for PII protection 
and finally ISO/IEC 29134 provides a structured \textbf{process for conducting Privacy Impact Assessments (PIAs)} to identify and mitigate privacy risks.

\vspace{\baselineskip}
Even if they originate from different ecosystem, with ISO being an \textbf{international standard} for global applicability
and NIST being a U.S. \textbf{federal standard}, all of them contribute to a comprehensive understanding of PII protection
and can be used together to enhance privacy practices within organizations. 

\vspace{\baselineskip}

\begin{figure}[h!]
\centering
\begin{tikzpicture}[
    node distance=2cm,
    every node/.style={rectangle, draw, rounded corners, align=center, minimum width=5cm, minimum height=1cm}
]

% Nodes
\node (iso29100) {ISO/IEC 29100\\\textit{Privacy Framework}};
\node (iso29134) [above=of iso29100] {ISO/IEC 29134\\\textit{Privacy Impact Assessment}};
\node (iso29151) [above=of iso29134] {ISO/IEC 29151\\\textit{PII Protection Controls}};

\node (nist) [right=4.5cm of iso29134] {NIST SP 800-122\\\textit{PII Confidentiality Safeguards}};

% Solid arrows (ISO structure)
\draw[->, thick] (iso29100) -- node[right=-1cm, draw=none]{\scriptsize provides principles\\\scriptsize and definitions} (iso29134);
\draw[->, thick] (iso29134) --  node[right=-1cm, draw=none]{\scriptsize provides results to\\\scriptsize select controls} (iso29151);

% Dashed arrows to NIST
\draw[->, thick, dashed] (nist.west) -- node[above, sloped, draw=none]{\scriptsize technical guidance} (iso29151.east);
\draw[->, thick, dashed] (nist.west) -- node[above=-0.1cm, sloped, draw=none]{\scriptsize complements} (iso29134.east);

\end{tikzpicture}
\caption{Relationship between ISO/IEC 29100, 29134, 29151 and NIST SP 800-122}
\end{figure}

To support a better understanding and comparison, the standards are examined based on a set of common
attributes including \textbf{objectives}, \textbf{scope} and \textbf{level of abstraction} at which they operate. Other comparison criteria
include the type of \textbf{output} each standard provides, the \textbf{methodological approach} they adopt (principle-base, risk-based, control-based),
and the various \textbf{roles} and \textbf{stakeholders} they address.

\subsection{Objectives}
The first aspect to consider is the \textbf{primary objectives} of each standard. 
ISO/IEC 29100 aims to establish
a \textbf{high-level framework} for PII protection by defining a \textbf{common terminology}, identifying the \textbf{actors}
that are involved in PII processing and outlining the fundamental \textbf{principles} for privacy protection.

\vspace{\baselineskip}
ISO/IEC 29134 has a different objective, focusing on providing a \textbf{structured methodology} for conducting
Privacy Impact Assessments (PIAs) focusing on the \textbf{identification}, \textbf{evaluation} and \textbf{treatment} of privacy risks
throughout the whole lifecycle of a process or system. 

\vspace{\baselineskip}
In contrast ISO/IEC 29151 translates the high-level principles
of ISO/IEC 29100 and the risk assessment results from ISO/IEC 29134 into \textbf{specific operational controls}
that organizations can implement to protect PII.

\vspace{\baselineskip}
Finally, NIST SP 800-122 serves several purposes:
it offers a \textbf{detailed and context-driven guide} to protecting the confidentiality of PII, introducing
\textbf{impact levels}, \textbf{risk factors} and \textbf{safeguard recommandations} that organizations can apply based on their specific needs.
Although their objectives differ significantly, these standards together cover the full spectrum of privacy governance.

\subsection{Methodological Approaches}
The standards also differ significantly in their \textbf{methodological approach}. 
ISO/IEC 29100 adopts a \textbf{principle-based methodology} by establishing fundamental privacy principles
that guide organizations in their PII protection efforts \textbf{without prescribing specific processes or controls}.
This means that it provides a broad framework that organizations can adapt to their specific context.

\vspace{\baselineskip}
ISO/IEC 29134, on the other hand, employs a \textbf{risk-based approach} by defining a \textbf{step-by-step procedure}
for conducting PIAs, including the identification of stakeholders, information flows, evaluation of the threats,
impacts and likelihoods and the selection of appropriate risk treatment options. 

\vspace{\baselineskip}
ISO/IEC 29151 takes a \textbf{control-based approach}, translating privacy risks and principles
into specific \textbf{technical security and privacy controls} aligned with the ISO/IEC 27002 framework.

\vspace{\baselineskip}
NIST SP 800-122 combines both \textbf{risk-based and control-based approaches} offering 
detailed criteria for determining PII confidentiality impact levels and recommending safeguards accordingly.

\vspace{\baselineskip}
These methodological differences illustrate how the standards operate at complementary layers, conceptual,
analytical, operational and technical, within a \textbf{unified privacy management ecosystem}.

\subsection{Scope and Regulatory Context}
Another important aspect to consider is the \textbf{scope} and \textbf{regulatory context} of each standard.
The ISO/IEC 29100 series is designed for \textbf{global applicability} and is intended to be \textbf{technology and jurisdiction-neutral},
allowing organizations of any size or sector to implement its principles and guidelines for a consistent
privacy management approach worldwide.
ISO/IEC 29001, 29134 and 29151 are commonly adopted by organizations seeking to \textbf{align with international best practices}
for privacy protection, such as GDPR, CCPA or other data protection regulations.

\vspace{\baselineskip}
NIST SP 800-122, in contrast, originates within the U.S. federal context and is primarily intended for \textbf{federal agencies}
and entities that handle PII on behalf of the U.S. government. Although it is tailored to the U.S. regulatory environment,
its guidelines and risk-based reccommendations are widely adopted internationally.

\vspace{\baselineskip}
These differences highlight how ISO standards aim to support broad governance and organizational compliance \textbf{across different legal contexts},
while NIST provides detailed operational guidance tailored to a specific regulatory context yet adaptable to broader use cases.

\subsection{Integration and Complementarity}
A key aspect emerging from this comparison is that the four standars are not intended to function in isolation,
but are instead \textbf{complementary components} of a comprehensive privacy management architecture. 
ISO/IEC 29100 establishes the foundational privacy principles and terminology that guide the interpretation and application of the other standards. 

\vspace{\baselineskip}
ISO/IEC 29134 builds directly on this foundation by providing a structured methodology to assess privacy risks and determine whether and how PII processing 
may impact individuals. 

\vspace{\baselineskip}
ISO/IEC 29151 then operationalizes the output of the PIA process by defining concrete controls 
and implementation guidance tailored to the risks identified through ISO/IEC 29134 and aligned with the principles of ISO/IEC 29100. 

\vspace{\baselineskip}
NIST SP 800-122 complements this ISO ecosystem by offering detailed technical safeguards and confidentiality impact assessment criteria that organizations 
can use to strengthen both their risk evaluation and their operational controls.

\vspace{\baselineskip}
When combined, these standards provide a comprehensive and layered approach that spans conceptual governance, analytical evaluation, 
operational implementation and technical protection.

\subsection{Critical Discussion}
Taken together, the four standards reveal a \textbf{complementary balance of strengths} but also \textbf{limitations} that organizations 
should take into account when adopting them. 

\vspace{\baselineskip}
ISO/IEC 29100 and ISO/IEC 29134 are very strong in 
providing a structured governance perspective (defining principles, roles and risk assessment methodologies) but they
leave implementation choises to organizations. ISO/IEC 29151 fills this gap by providing concrete controls, but
its effectiveness depends on the quality of the underlying risk assessment process and the organization's ability
to adapt controls to their specific context. 

\vspace{\baselineskip}
NIST SP 800-122 excels for its practical orientation and detailed guidance 
on condifentiality safeguards, but its focus is narrower than that of the ISO standards and primarly centered on U.S. federal environment.

\vspace{\baselineskip}
Despite these limitations, the standards are most effective when used in combination: 
ISO provides the governance and \textbf{methodological backbone} while NIST enriches the framework with \textbf{actionable technical criteria}. 
This layered approach enables organizations to build privacy programs that are conceptually robust, risk-aware and operationally sound, 
while maintaining the flexibility needed to adapt to varying legal and technical contexts.

\begin{table}[h!]
    \centering
    % chktex-file 44
    \begin{tabular}{|>{\raggedright\arraybackslash}p{3.5cm}|
                >{\raggedright\arraybackslash}p{3.5cm}|
                >{\raggedright\arraybackslash}p{3.5cm}|
                >{\raggedright\arraybackslash}p{3.5cm}|}
    \hline
    \textbf{Dimension} &
    \textbf{ISO/IEC 29100} &
    \textbf{ISO/IEC 29134} &
    \textbf{ISO/IEC 29151} \\
    \hline

    \textbf{Primary Objective} &
    Define privacy framework, principles and roles &
    Provide methodology for Privacy Impact Assessment (PIA) &
    Define operational controls for PII protection \\ 
    \hline

    \textbf{Approach} &
    Principle-based, conceptual &
    Risk-based, methodological &
    Control-oriented, implementation-driven \\
    \hline

    \textbf{Output} &
    Privacy principles, terminology, actors &
    PIA report, risk evaluation, treatment plan &
    Set of privacy and security controls aligned with ISO/IEC 27002 \\
    \hline

    \textbf{Scope of Application} &
    All organizations processing PII &
    Projects, systems or processes involving PII &
    Organizations implementing operational privacy safeguards \\
    \hline

    \textbf{Role in Privacy Management} &
    Foundational layer: defines “what” privacy requires &
    Analytical layer: evaluates “how” PII processing affects individuals &
    Operational layer: defines “how” risks must be mitigated \\
    \hline
    \end{tabular}

\vspace{0.5cm}

\renewcommand{\arraystretch}{1.4}
\begin{tabular}{|>{\raggedright\arraybackslash}p{3.5cm}|p{10.5cm}|}
\hline
\textbf{NIST SP 800-122} &
Provides technical guidance for protecting the confidentiality of PII, defining impact levels, risk factors and recommended safeguards. Complements ISO 29134 and ISO 29151 with detailed technical criteria. \\
\hline
\end{tabular}

\caption{Comparative overview of ISO/IEC 29100, 29134, 29151 and NIST SP 800-122}
\end{table}
