%section
\subsection{ISO/IEC 29151: Overview of the standard and Purpose}
ISO/IEC 29151 is a globally recognised standard that provides guidelines for protecting personally identifiable information (PII) in information technology.
The primary focus of ISO 29151 is to establish controls and guidelines that organisations can implement to manage privacy risks related to the processing of PII.

\vspace{\baselineskip}
ISO/IEC 29151 is a crucial component of data privacy as it provides a comprehensive framework for the protection of personal data. It is particularly relevant today, as the collection, processing, and storage of personal data have become commonplace. This standard is designed to help organisations manage the privacy of personal data in a systematic and consistent manner, thereby reducing the risk of data breaches and ensuring compliance with data protection regulations.

\vspace{\baselineskip}
The standard applies to all types and sizes of organisations, including public and private companies, government entities, and not-for-profit organisations. It is also applicable to all sectors and industries that process PII, such as healthcare, finance, education, and retail. 

In general, it is applicable to all types and sizes of organizations acting as PII controllers, as defined in ISO/IEC 29100.
The standard is designed to be flexible and can be tailored to each organisation's specific needs and circumstances.



\vspace{\baselineskip}
While implementing ISO/IEC 29151 can provide several benefits, it can also pose some challenges. One of the main challenges is the complexity of the standard, which requires a deep understanding of privacy principles and controls. Implementing the standard can also require significant time and resources, particularly for organisations that do not have a pre-existing privacy management system.

Another challenge in implementing ISO 29151 is the need for cultural change. Protecting PII requires the involvement and commitment of all employees, which can be difficult to achieve in organisations with a low awareness of privacy issues. Furthermore, the standard requires ongoing monitoring and review, which can be challenging to maintain over time. \cite{privacyengine_iso_29151_2025}

It was published on September 28, 2017 and reviewed in 2022.

\begin{figure}[H] 
    \centering
    \includegraphics[width=0.5\textwidth]{images/iso_29151.png} % percorso e scala
    \caption{ISO/IEC 29151}
    \label{fig:iso29151}
\end{figure}

%section
\subsection{Requirments for the Protection of PII}
An organization should identify its PII protection requirements and the privacy principles in ISO/IEC 29100 apply to the identification of requirements.

There are three main sources of PII protection requirements:
\begin{itemize}
    \item legal, statutory, regulatory and contractual requirements related to protection of PII;
    \item assessment of risks (i.e., security risks and privacy risks) to the organization and the PII principal, taking into account the organization’s overall business strategy and objectives, through a risk assessment; 
    \item corporate policies: an organization may also choose voluntarily to go beyond the criteria that are derived from previous requirements.
\end{itemize}

PII protection controls should be selected on the basis of a risk assessment. The results of a 
privacy impact assessment (PIA) (later analysed in ISO/IEC 29134) will help to guide and determine the appropriate 
treatment action and priorities for managing risks to the protection of PII and for implementing controls selected to protect 
against these risks.

%section
\subsection{Structure of the Standard}
It is structured into several sections, each of which covers a specific aspect of privacy controls. The main sections of the standard include the scope, normative references, terms and definitions, privacy principles, and controls.

The standard also includes several annexes that provide additional guidance and examples of privacy controls and cover topics such as the relationship between ISO/IEC 29151 and other standards, the implementation of privacy controls, and the management of privacy risks.


\vspace{\baselineskip}
The controls presented in this standard can be used as reference for organizations that process PII, and the selection of controls  is dependent upon organizational decisions based on the criteria for risk treatment options. 
If required, controls can also be selected from other control sets or new controls can be designed to meet specific needs, as appropriate. 

\vspace{\baselineskip}
The following sections of this report provide a detailed analysis of privacy controls and guidelines defined in \textbf{ISO/IEC 29151}, by following and controls defind in ISO/IEC 27002.
The format for this part uses the relevant clause headings and numbering from ISO/IEC 27002 to allow cross-reference to that International Standard. 
For these reason, in this report, some references to ISO/IEC 27002 will be present as well.


%section
\subsection{Information Security Policies --> CITA 27002}
\subsubsection{Objective and Implementation guidance}
The objective of this control is to

"\textit{provide management direction and support for information security in accordance with 
business requirements and relevant laws and regulations.}"

\vspace{\baselineskip}
Organizations should define an information security policy which sets out the organisation's approach to managing its information security objectives.

At lower level, the information security policy should be supported by policies related to specific topics, such as access control, physical and encironmental security, backup, cryptography, and so on.
These policies should be communicated to employees and relevant external parties in a form that is 
relevant, accessible and understandable to the intended reader.

\subsubsection{Implementation guidance for the protection of PII}
This section is present in ISO/IEC 29151 for every control analysed, in addition to the implementation guidance provided in ISO/IEC 27002.
It is used in order to highlight in which way the control evolved to specifically address the protection of PII.

\vspace{\baselineskip}
For instance, for this specific control, ISO/IEC 29151 suggests that the information security policy should include specific aspects related to PII protection. 
These aspects are described in the Annex 2 of the standard and include some guidance already mentioned in ISO 29100, such as:
\begin{itemize}
    \item the PII protection policy should be appropriate to the purpose of the organization;
    \item it should define rules for making decisions in questions of protection of PII;
    \item it should include a commitment to continual improvement;
    \item it should be be transparent about the organization's collection and processing of PII.
\end{itemize}

%section
\subsection{Organization of Information Security - Internal Organization}
\subsubsection{Objective and Implementation guidance}
The objective in this case is the following:

"\textit{To establish a management framework to initiate and control the implementation and 
operation of information security within the organization.}"
\vspace{\baselineskip}

This control is divided into several sub-controls that cover different aspects of the internal organizations. The following sub-sections are the most relevant ones:

\vspace{\baselineskip}
\textbf{Information security roles and responsabilities}:
\\
Allocation of information security responsibilities should be done in accordance with the information security policies previously defined.
It is important to define local responsabilities for the protection of assets and for information security risk management.

Many organisations rely on an information security manager to take overall responsibility for the 
development and implementation of information security and to support the identification of controls.

However, responsibility for resourcing and implementing the controls will often remain with individual 
managers. One common practice is to appoint an owner for each asset who then becomes responsible 
for its day-to-day protection.

\vspace{\baselineskip}
\textbf{Segregation of duties}:
\\
This sub-control highlights the importance of reducing opportunities for unauthorized or unintentional modification or misuse of the organization's assets, by means of segregation of duties.
It is important to guarantee that each individual has proper authorization for the access and use of assets.

It may be difficult for small organizations to achieve segregation of duties, but they need to apply the control as far as it is possibile and eventually considering other controls.

\vspace{\baselineskip}
\textbf{Contact with authorities}:
\\
One relevant aspect of this sub-control is the need to maintain contact with relevant authorities. 
Organizations should have procedures in place that specify when and by whom authorities should be contacted and how identified 
information security incidents should be reported in a timely manner.

Contacts with regulatory bodies are also useful to anticipate and prepare for upcoming changes in laws or 
regulations, which have to be implemented by the organization.

\subsubsection{Implementation guidance for the protection of PII}
These controls are amplified in ISO/IEC 29151 with specific implementation guidance for the protection of PII.

In particular, regarding role and responsabilities, it is important to define specific roles for the protection of PII, such as an individual or a group of them that should have the responsability for guarantee the coordination between information security functions and PII protection functions.

The organisation should consider the protection of PII as a multi-disciplinary activity, with different groups that can collaborate to reach the same goal. 
For instance, there may be a group that identify new risks and areas for conducting PIAs, planning preventive actions, and studying detection measures for any breaches. 

At the end, all these groups should be coordinated by an individual responsible for PII protection.
So, these groups can be considered as PII processors that follow the instructions of a PII controller, as defined in ISO/IEC 29100.

\vspace{\baselineskip}
With respect to Segragation of Duties, a fondamental aspect to consider is that duties related to PII protection should be independent of those for information security. 
As a consequence, the principle of segragation of duties should be considered when dealing with access rights for PII processing, especially when the risk involved in processing PII is high.

\vspace{\baselineskip}
Finally, regarding Contact with Authorities, organizations should have procedures that specify when and by whom authorities, such as Data Protection Authorities, should be contacted to report privacy breaches or to report processing details.

%section
\subsection{Human Resource Security}
\subsubsection{Objective and Implementation guidance for the protection of PII}
The objective of this control is to:

\vspace{\baselineskip}
"\textit{to ensure that employees and contractors understand their responsibilities and are suit
able for the roles for which they are considered, and to  ensure that employees and contractors are aware of and fulfil their information security responsibilities.}"

\vspace{\baselineskip}

This section of the standard contains many controls related to Human Resource Security, that are fundamental for the protection of PII.
In particular, this report focuses only on the controls that ISO/IEC 29151 amplifies with specific implementation guidance for the protection of PII.

One of them is \textbf{Human resource security during empolyment}. 
It is important to guarantee that employees:
\begin{itemize}
    \item apply information security in accordance with the established policies of the organization;
    \item achieve a level of awerness relevant to their roles and responsabilities;
    \item have the necessary skills and continue to develop them.
\end{itemize}

If they do not respect these aspects, they could represent a risk for the organization and for the protection of PII.
With respect to PII, one fundamental aspect introduced with ISO/IEC 29151 is the need to provide specific training for the possible consequences for the PII controller
and for PII principals, such as legal consequences or reputational damage. As a consequence, employees should be trained in the appropriate way to handle PII.

%section
\subsection{Asset Management - Responsibility for Assets and Information Classification}
\subsubsection{Objective and Implementation guidance for the protection of PII}

In this case, the two main controls analysed with respect to Asset Management are the \textbf{Responsability for Assets} and \textbf{Information Classification}.
Respectively, their objectives are the following:

\vspace{\baselineskip}
"\textit{to identify organizational assets and define appropriate protection responsabilities.}"

"\textit{to  ensure that information receives an appropriate level of protection in accordance with 
its importance to the organization.}"

\vspace{\baselineskip}
With respect to \textbf{Responsability for assets}, ISO/IEC 27002 suggests that all assets should be clearly identified and an organisation should document their importance.
The asset inventory should be accurate, up to date, consistent and aligned with other inventories.

Regarding guidance for PII protection, ISO/IEC 29151 suggests that this invetory of assets should be established and maintained following the information reported in the PIA report, as it is later analysed in \textbf{ISO/IEC 29134}.
Starting from the PIA report, the organisation should extract some relevant informations, such as the types of PII processed, the level of impact of any breach of PII, the purposes of collecting the PII and the retention period of PII.

\vspace{\baselineskip}
With respect to \textbf{Information Classification}, it is suggested to classify general information according to legal requirements and criticality and sensitivity to unauthorised disclosure or modification.
The classification scheme should be consistent across the whole organization so that everyone will classify information 
and related assets in the same way, have a common understanding of protection requirements and apply the appropriate protection.

\vspace{\baselineskip}
For PII protection, ISO/IEC 29151 shows that the classification scheme should consider also all information containing PII.
For instance, may also include more specific categories, such as personal health information or personal financial information.

Some PII that may be classified non-sensitive in one country may be treated as sensitive elsewhere, depending on the 
applicable data protection laws, so as a consequence, the  classification for an element of PII could need re-evaluation and modification when associated with one or more additional attributes.

%section
\subsection{Access Control - User Access Management}
\subsubsection{Objective and Implementation guidance for the protection of PII}
This control is focused on:

\vspace{\baselineskip}
"\textit{ensure authorized user access and prevent unauthorized access to systems and services.}"

\vspace{\baselineskip}
This means that procedures for user registration and de-registration should be implemented and provide also measures to manage a compromise of user access control, such as 
the corruption or compromise of passwords or other user registration data.

It is also important that organizations provide users with an appropriate right of access to the information systems processing PII, in 
accordance with the data minimization principle described in ISO/IEC 29100. 
It is necessary to limit access to information systems processing PII to only those users who really need that access. As a consequence, strong authentication methods should be applied for particular PII and PII processing.

ISO/IEC 29151 focuses its attention also on the fact that large scale processing of PII can increase the risk of large scale breaches.
Therefore, it is necessary to take special care when dealig with access rights for these privileged operations.
The e granting and use of such rights should be recorded in relevant log files and regularly reviewed.


%section
\subsection{Cryptography}
\subsubsection{Objective and Implementation guidance for the protection of PII}
The objective in this case is:

\vspace{\baselineskip}
"\textit{to ensure proper and effective use of cryptography to protect the confidentiality, authenticity and/or integrity of information.}"

\vspace{\baselineskip}
For this specific section of the standard, no specific implementation guidance for the protection of PII is provided. 
It is considered that the general guidance provided in ISO/IEC 27002 is sufficient to address the protection of PII as well.

The following are some relevant aspects regarding Cryptography that are mentioned in ISO/IEC 27002 and automatically apply to PII protection.
 When developing a cryptographic policy, the most important aspects to consider are the following:
\begin{itemize}
   \item based on a risk assessment, the required level of protection should be identified taking into account the type, strength and quality of the encryption algorithm required;
   \item who is responsible for implement the policy and for managing the cryptographic keys;
   \item the standards to be adopted for effective implementation throughout the organization;
   \item the regulations and national restrictions that might apply to the use of cryptographic techniques in different parts of the world;
   \item requirments for key management, such as key generation, distribution, storage, use, destruction and archiving;
   \item protection of cryptographic keys against modification, loss or destruction, and against unauthorised access.
\end{itemize}

These are only the most relevant aspects mentioned in ISO/IEC 27002 and all of them can be applied to PII protection as well.

%section
\subsection{Physical and Environmental Security - Secure Areas and Equipment}
\subsubsection{Objective and Implementation guidance for the protection of PII}
The objective of this control is:

\vspace{\baselineskip}
"\textit{to prevent unauthorized physical access, damage and interference to the organization’s information
and information processing facilities.}"

\vspace{\baselineskip}
This section of the standard contains many controls related to Physical and Environmental Security, such as supporting utilities, cabling security, equipment maintenance, secure disposal or re-use of equipement, and so on.
They express some important concepts that have important roles in the Environmental Security.
The following are the most relevant ones with respect to the subsection of \textbf{Secure Areas}:
\begin{itemize}
    \item \textbf{Physical Security Perimeter}: define perimeters to protect areas that contain sensitive and PII informations;
    \item \textbf{Physical Entry Controls}: implement proper entry controls to prevent unauthorised access to secure areas;
    \item \textbf{Protecting against External and Environmental Threats}: implement proper measures to protect secure areas from external and environmental threats, such as natural disasters, criminal activity, and so on;
    \item \textbf{Working in Secure Areas}: implement proper procedures for employees and contractors working in secure areas, such as access control and supervision.
\end{itemize}

\vspace{\baselineskip}
With respect to \textbf{Equipment}, the most relevant aspects are the following:
\begin{itemize}
    \item \textbf{Equipment Siting and Protection}: proper siting and protection of equipment to reduce the risks from environmental threats;
    \item \textbf{Cabling Security}: proper protection of cabling carrying sensitive or PII information from interception, tapping or damage;
    \item \textbf{Equipment Maintenance}: proper maintenance of equipment to ensure its continued availability and integrity;
    \item \textbf{Secure Disposal or Re-use of Equipement}: proper procedures for the secure disposal or re-use of equipment containing sensitive or PII information, to prevent unauthorised access or disclosure.
\end{itemize}

\vspace{\baselineskip}
In ISO/IEC 29151, \textbf{Secure Disposal or Re-use of Equipement} is the only one amplified with specific implementation guidance for the protection of PII.
It specifies that equipment containing storage media that may possible contai PII should be destroyed or the PII should either be destroyed, deleted or overwritten using approved techniques,
to render the original PII unrecoverable rather than simply using the standard delete function.
For equipment containing storage media that may possibly contain encrypted 
PII, the controlled destruction of decryption keys or key holders (such as smart cards), may be sufficient. 

%section
\subsection{Operations Security}
\subsubsection{Objective and Implementation guidance for the protection of PII}
Objectives:

\vspace{\baselineskip}
"\textit{to ensure correct and secure operation of information processing facilities;
to protect against loss of data;
to record events and generate evidence.}"

\vspace{\baselineskip}
Also in this case different are the controls analysed in this section of the standard.
By focusing only on that ones that are amplified by ISO/IEC 29151, it is possible to identify \textbf{Operational Procedures and Responsabilities},
\textbf{Backup} and \textbf{Logging and Monitoring}.

\vspace{\baselineskip}
\textbf{Operational Procedures and Responsabilities}:
\\
Operating procedures should be documented and made available to all users who need them.
These includes the installation and configuration of systems, instructions for handling errors, system restart and recovery procedures for use in the event of system failure.

\vspace{\baselineskip}
Specific to PII protection, ISO/IEC 29151 suggests that development, testing and operational environments should be logically and, where possible, physically separate environments.
Appropriate access controls should be implemented to ensure access is limited to properly authorized individuals.
Where not permitted by law or by explicit consent of the PII principal, PII should not be used for purposes of development and testing without prior anonymization. 

\vspace{\baselineskip}
\textbf{Backup}:
\\
Backup copies of information, software and system images should be taken and tested regularly in accordance with an agreed backup policy.
Adequate backup facilities should be provided to ensure that all essential information and software can be recovered following a disaster or media failure.

Information systems processing PII should introduce additional or alternative mechanisms, such as off-site backups for 
protection against loss of PII, ensuring continuity of PII processing operations, and providing the ability to restore 
PII processing operations after a disruptive event, if only strictly necessary. 

\vspace{\baselineskip}
\textbf{Logging and Monitoring}:
\\
Event logs recording user activities, exceptions, faults and information security events should be produced, kept and regularly reviewed.
Event logs should include userIDs, system activities, dates, times and details of key events, use of privileges, records of successful and rejected system access attempts, and so on.

It is important also to protect log information against tampering and unauthorized access, to ensure its integrity.

\vspace{\baselineskip}
For PII protection, where possible, the event log should record which PII was accessed, what was done to the PII, especially for certain types of PII like health data.
In addition, the PII controller should define procedures regarding whether, when and how log information can be made available to 
or usable by the administrator for purposes such as security monitoring and operational diagnostics. 

%section
\subsection{Communications Security - Information Transfer}
\subsubsection{Objective and Implementation guidance for the protection of PII}
The objective is:

\vspace{\baselineskip}
"\textit{to maintain the security of information transferred within an organization and with any 
external entity.}"

\vspace{\baselineskip}
This control is introduced in order to ensure that formal \textbf{Transfer Policies and Procedures} are in place to protect the transfer of information.
It focuses also on requirements for \textbf{Confidentiality or Non-disclosure Agreements}, that need to reflect the organization's needs for the protection of information and should be identified, reguarly reviewed and documented.

With respect to PII protection, ISO/IEC 29151 suggests to apply some appropriate measures to reduce the risk of PII leakage during information transfer.
This can be solved by using encryption or other measures like de-identification, masking or obfuscation.
For confidentiality or non-disclosure agreements, it is important to include the conditions under which external processing of PII may take place, and these conditions 
should be part of an appropriate agreement (e.g., contract, confidentiality or non-disclosure agreement). 

%section
\subsection{System Acquisition, Development and Maintenance}
This control contains two main sub-controls that ISO/IEC 29151 amplifies: \textbf{Security Requirements of Information Systems} and \textbf{Test Data}.
Their objectives are the following:

\vspace{\baselineskip}
"\textit{to ensure that information security is an integral part of information systems across the 
entire lifecycle. This also includes the requirements for information systems which provide services 
over public networks}"

"\textit{to ensure the protection of data used for testing.}"

\vspace{\baselineskip}
\textbf{Security Requirements of Information Systems}:
\\
The information security related requirements should be included in the requirements for new 
information systems or enhancements to existing information systems.
Identification and management of information security requirements and associated processes should 
be integrated in early stages of information systems projects. Early consideration of information security 
requirements, e.g. at the design stage can lead to more effective and cost efficient solutions.

For protection of PII, ISO/IEC 29151 suggests that when developing or making significant changes to information systems that process PII, a PIA should be conducted. 
The results of the PIA should be used to determine the 
controls to treat the risks identified during the PIA process.
Guidance for conducting PIAs is provided in ISO/IEC 29134, in the next chapter of this report.

\vspace{\baselineskip}
\textbf{Test Data}:
\\
Test data should be selected carefully, protected and controlled.
Operational data containing PII should not normally be used for development and testing. The use of real PII in these 
environments increases the risk of information compromise. Instead, organizations should either use synthetic data or 
should take steps to ''hide'' (e.g., mask, obfuscate, de-identify) any real PII in use. 

%section
\subsection{Supplier Relationships}
Supplier Relationships control focus on the following objective:

\vspace{\baselineskip}
"\textit{to ensure protection of the organization's assets that is accessible by suppliers.}"

\vspace{\baselineskip}
This section of the standard contains one main sub-control amplified by ISO/IEC 29151: \textbf{Information Security in Supplier Relationships}.
It refers to the fact that information security requirements for mitigating the risks associated with supplier’s access to the 
organization’s assets should be agreed with the supplier and documented.

In the event that an organization needs to make use of the services of a PII processor, PII processors should be evaluated 
on the basis of experience, trustworthiness and their ability to meet PII protection requirements.
The organization acting as a PII controller should have a written contract with any supplier acting as a PII processor. 
The contract should clearly allocate roles and responsibilities between the PII controller and the PII processor and should 
contain appropriate clauses relating to PII protection in order to hold the PII processor accountable for the processing performed.

The PII controller should ensure that their PII processors do not process the PII for any purposes other than those specified 
in the contract or other legal agreement and that the PII processor securely dispose of PII.

%section
\subsection{Information Security Incident Management}
The main control is \textbf{Management of Information Security Incident and Improvements} and it is focused on:

\vspace{\baselineskip}
"\textit{ensure a consistent and effective approach to the management of information security 
incidents, including communication on security events and weaknesses.}"

\vspace{\baselineskip}

The two important sub-controls amplified in ISO/IEC 29151 are the following:
\begin{itemize}
    \item \textbf{Responsabilities and Procedures};
    \item \textbf{Reporting Information Security Events}.
\end{itemize}

\vspace{\baselineskip}
\textbf{Responsabilities and Procedures}:
\\ 
Management responsibilities and procedures should be established to ensure a quick, effective and 
orderly response to information security incidents.

Organizations should be capable of providing an effective response to a privacy incident and implement a privacy incident response plan.
The privacy plan should include:
\begin{itemize}
    \item the definition of privacy incident and the scope of privacy incident response;
    \item the establishment of a cross-functional privacy incident response team that develops, implements, tests, executes and reviews the privacy incident response plan;
    \item clearly defined roles, responsibilities and authorities for all members of the privacy incident response team; 
    \item an incident impact assessment  to determine the nature of any potential or actual harms to affected individuals;
    \item procedures to determine whether notice to affected individuals and other designated entities is required, the timing for such notice and the form used to provide that notice.
\end{itemize}

\vspace{\baselineskip}
Organizations may choose to integrate their privacy incident response plans with their security incident response plans or keep them separate. 
An information security incident should trigger a review by the PII controller to determine if a data breach involving PII has taken place.

\vspace{\baselineskip}
\textbf{Reporting Information Security Events}:
\\
Information security events should be reported through appropriate management channels as quickly as possible.

When PII is compromised, the rights and interests of the PII principal cannot be protected without immediate measures. 
Jurisdictions may impose specific requirements related to the reporting or notification of security incidents involving PII.
When a security incident related to PII occurs, the details of the incident should be notified as soon as possible to the relevant authorities.

Organizations should provide affected PII principals access to appropriate and effective remedies, such as correction or 
deletion of incorrect information, if a privacy breach has occurred. 

%section
\subsection{Compliance}
\textbf{Compliance with Legal and Contractual Requirements}
\\
The objective is:

\vspace{\baselineskip}
"\textit{to avoid breaches of legal, statutory, regulatory or contractual obligations related to information security and of any security requirements.}"

\vspace{\baselineskip}
All relevant legislative statutory, regulatory, contractual requirements and the organization’s approach 
to meet these requirements should be explicitly identified, documented and kept up to date for each 
information system and the organization.

Organizations should identify the laws and regulations related to PII protection to which they are subject. If these are 
identified, then organizations should take necessary measures for those requirements.

Organizations should develop PIAs and implement the resulting privacy treatment plans in order to help ensure that 
programmes and services related to PII processing comply with privacy safeguarding requirements. Further guidance will 
be found in ISO/IEC 29134. 

They should also establish an audit programme to help verify that PII processing complies with relevant privacy 
safeguarding requirements. The programme should specify the frequency with which audits are to be conducted.

\vspace{\baselineskip}
While in many jurisdictions it will be the PII controller who is ultimately responsible for ensuring compliance, all actors 
involved in the processing of PII should take a proactive approach in identifying relevant privacy safeguarding 
requirements arising from legal or other factors.

\vspace{\baselineskip}
\textbf{Information Security Reviews}
\\
The objective is:

\vspace{\baselineskip}
"\textit{to ensure that information security is implemented and operated in accordance with the 
organizational policies and procedures.}"

\vspace{\baselineskip}
The organization’s approach to managing information security and its implementation should be reviewed 
independently at planned intervals or when significant changes occur.

If audits by individual interested parties are impractical or may increase risks to security, organizations should make 
available to prospective interested parties , independent evidence that information security 
is implemented and operated in accordance with the PII controller's policies and procedures.
