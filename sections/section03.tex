%section
\subsection{ISO/IEC 29151: Overview of the standard and Purpose}
ISO/IEC 29151 is a globally recognised standard that provides guidelines for protecting personally identifiable information (PII) in information technology.
The primary focus of ISO 29151 is to establish controls and guidelines that organisations can implement to manage privacy risks related to the processing of PII.

\vspace{\baselineskip}
ISO/IEC 29151 is a crucial component of data privacy as it provides a comprehensive framework for the protection of personal data. It is particularly relevant today, as the collection, processing, and storage of personal data have become commonplace. This standard is designed to help organisations manage the privacy of personal data in a systematic and consistent manner, thereby reducing the risk of data breaches and ensuring compliance with data protection regulations.

\vspace{\baselineskip}
The standard applies to all types and sizes of organisations, including public and private companies, government entities, and not-for-profit organisations. It is also applicable to all sectors and industries that process PII, such as healthcare, finance, education, and retail. 

In general, it is applicable to all types and sizes of organizations acting as PII controllers, as defined in ISO/IEC 29100.
The standard is designed to be flexible and can be tailored to each organisation's specific needs and circumstances.



\vspace{\baselineskip}
While implementing ISO/IEC 29151 can provide several benefits, it can also pose some challenges. One of the main challenges is the complexity of the standard, which requires a deep understanding of privacy principles and controls. Implementing the standard can also require significant time and resources, particularly for organisations that do not have a pre-existing privacy management system.

Another challenge in implementing ISO 29151 is the need for cultural change. Protecting PII requires the involvement and commitment of all employees, which can be difficult to achieve in organisations with a low awareness of privacy issues. Furthermore, the standard requires ongoing monitoring and review, which can be challenging to maintain over time. \cite{privacyengine_iso_29151_2025}

It was published on September 28, 2017 and reviewed in 2022.

\begin{figure}[H] 
    \centering
    \includegraphics[width=0.5\textwidth]{images/iso_29151.png} % percorso e scala
    \caption{ISO/IEC 29151}
    \label{fig:iso29151}
\end{figure}

%section
\subsection{Requirments for the Protection of PII}
An organization should identify its PII protection requirements and the privacy principles in ISO/IEC 29100 apply to the identification of requirements.

There are three main sources of PII protection requirements:
\begin{itemize}
    \item legal, statutory, regulatory and contractual requirements related to protection of PII;
    \item assessment of risks (i.e., security risks and privacy risks) to the organization and the PII principal, taking into account the organization’s overall business strategy and objectives, through a risk assessment; 
    \item corporate policies: an organization may also choose voluntarily to go beyond the criteria that are derived from previous requirements.
\end{itemize}

PII protection controls should be selected on the basis of a risk assessment. The results of a 
privacy impact assessment (PIA) (later analysed in ISO/IEC 29134) will help to guide and determine the appropriate 
treatment action and priorities for managing risks to the protection of PII and for implementing controls selected to protect 
against these risks.

%section
\subsection{Structure of the Standard}
It is structured into several sections, each of which covers a specific aspect of privacy controls. The main sections of the standard include the scope, normative references, terms and definitions, privacy principles, and controls.

The standard also includes several annexes that provide additional guidance and examples of privacy controls and cover topics such as the relationship between ISO/IEC 29151 and other standards, the implementation of privacy controls, and the management of privacy risks.


\vspace{\baselineskip}
The controls presented in this standard can be used as reference for organizations that process PII, and the selection of controls  is dependent upon organizational decisions based on the criteria for risk treatment options. 
If required, controls can also be selected from other control sets or new controls can be designed to meet specific needs, as appropriate. 

\vspace{\baselineskip}
The following sections of this report provide a detailed analysis of privacy controls and guidelines defined in \textbf{ISO/IEC 29151}, by following and controls defind in ISO/IEC 27002.
The format for this part uses the relevant clause headings and numbering from ISO/IEC 27002 to allow cross-reference to that International Standard. 
For these reason, in this report, some references to ISO/IEC 27002 will be present as well.


%section
\subsection{Information Security Policies --> CITA 27002}
\subsubsection{Objective and Implementation guidance}
The objective of this control is to

"\textit{provide management direction and support for information security in accordance with 
business requirements and relevant laws and regulations.}"

\vspace{\baselineskip}
Organizations should define an information security policy which sets out the organisation's approach to managing its information security objectives.

At lower level, the information security policy should be supported by policies related to specific topics, such as access control, physical and encironmental security, backup, cryptography, and so on.
These policies should be communicated to employees and relevant external parties in a form that is 
relevant, accessible and understandable to the intended reader.

\subsubsection{Implementation guidance for the protection of PII}
This section is present in ISO/IEC 29151 for every control analysed, in addition to the implementation guidance provided in ISO/IEC 27002.
It is used in order to highlight in which way the control evolved to specifically address the protection of PII.

\vspace{\baselineskip}
For instance, for this specific control, ISO/IEC 29151 suggests that the information security policy should include specific aspects related to PII protection. 
These aspects are described in the Annex 2 of the standard and include some guidance already mentioned in ISO 29100, such as:
\begin{itemize}
    \item the PII protection policy should be appropriate to the purpose of the organization;
    \item it should define rules for making decisions in questions of protection of PII;
    \item it should include a commitment to continual improvement;
    \item it should be be transparent about the organization's collection and processing of PII.
\end{itemize}

%section
\subsection{Organization of Information Security - Internal Organization}
\subsubsection{Objective and Implementation guidance}
The objective in this case is the following:

"\textit{To establish a management framework to initiate and control the implementation and 
operation of information security within the organization.}"
\vspace{\baselineskip}

This control is divided into several sub-controls that cover different aspects of the internal organizations. The following sub-sections are the most relevant ones:

\vspace{\baselineskip}
\textbf{Information security roles and responsabilities}:
\\
Allocation of information security responsibilities should be done in accordance with the information security policies previously defined.
It is important to define local responsabilities for the protection of assets and for information security risk management.

Many organisations rely on an information security manager to take overall responsibility for the 
development and implementation of information security and to support the identification of controls.

However, responsibility for resourcing and implementing the controls will often remain with individual 
managers. One common practice is to appoint an owner for each asset who then becomes responsible 
for its day-to-day protection.

\vspace{\baselineskip}
\textbf{Segregation of duties}:
\\
This sub-control highlights the importance of reducing opportunities for unauthorized or unintentional modification or misuse of the organization's assets, by means of segregation of duties.
It is important to guarantee that each individual has proper authorization for the access and use of assets.

It may be difficult for small organizations to achieve segregation of duties, but they need to apply the control as far as it is possibile and eventually considering other controls.

\vspace{\baselineskip}
\textbf{Contact with authorities}:
\\
One relevant aspect of this sub-control is the need to maintain contact with relevant authorities. 
Organizations should have procedures in place that specify when and by whom authorities should be contacted and how identified 
information security incidents should be reported in a timely manner.

Contacts with regulatory bodies are also useful to anticipate and prepare for upcoming changes in laws or 
regulations, which have to be implemented by the organization.

\subsubsection{Implementation guidance for the protection of PII}
These controls are amplified in ISO/IEC 29151 with specific implementation guidance for the protection of PII.

In particular, regarding role and responsabilities, it is important to define specific roles for the protection of PII, such as an individual or a group of them that should have the responsability for guarantee the coordination between information security functions and PII protection functions.

The organisation should consider the protection of PII as a multi-disciplinary activity, with different groups that can collaborate to reach the same goal. 
For instance, there may be a group that identify new risks and areas for conducting PIAs, planning preventive actions, and studying detection measures for any breaches. 

At the end, all these groups should be coordinated by an individual responsible for PII protection.
So, these groups can be considered as PII processors that follow the instructions of a PII controller, as defined in ISO/IEC 29100.

\vspace{\baselineskip}
With respect to Segragation of Duties, a fondamental aspect to consider is that duties related to PII protection should be independent of those for information security. 
As a consequence, the principle of segragation of duties should be considered when dealing with access rights for PII processing, especially when the risk involved in processing PII is high.

\vspace{\baselineskip}
Finally, regarding Contact with Authorities, organizations should have procedures that specify when and by whom authorities, such as Data Protection Authorities, should be contacted to report privacy breaches or to report processing details.

%section
\subsection{Human Resource Security}
\subsubsection{Objective and Implementation guidance for the protection of PII}
The objective of this control is to:

\vspace{\baselineskip}
"\textit{to ensure that employees and contractors understand their responsibilities and are suit
able for the roles for which they are considered, and to  ensure that employees and contractors are aware of and fulfil their information security responsibilities.}"

\vspace{\baselineskip}

This section of the standard contains many controls related to Human Resource Security, that are fundamental for the protection of PII.
In particular, this report focuses only on the controls that ISO/IEC 29151 amplifies with specific implementation guidance for the protection of PII.

One of them is \textbf{Human resource security during empolyment}. 
It is important to guarantee that employees:
\begin{itemize}
    \item apply information security in accordance with the established policies of the organization;
    \item achieve a level of awerness relevant to their roles and responsabilities;
    \item have the necessary skills and continue to develop them.
\end{itemize}

If they do not respect these aspects, they could represent a risk for the organization and for the protection of PII.
With respect to PII, one fundamental aspect introduced with ISO/IEC 29151 is the need to provide specific training for the possible consequences for the PII controller
and for PII principals, such as legal consequences or reputational damage. As a consequence, employees should be trained in the appropriate way to handle PII.

%section
\subsection{Asset Management - Responsibility for Assets and Information Classification}
\subsubsection{Objective and Implementation guidance for the protection of PII}

In this case, the two main controls analysed with respect to Asset Management are the \textbf{Responsability for Assets} and \textbf{Information Classification}.
Respectively, their objectives are the following:

\vspace{\baselineskip}
"\textit{to identify organizational assets and define appropriate protection responsabilities.}"

"\textit{to  ensure that information receives an appropriate level of protection in accordance with 
its importance to the organization.}"

\vspace{\baselineskip}
With respect to \textbf{Responsability for assets}, ISO/IEC 27002 suggests that all assets should be clearly identified and an organisation should document their importance.
The asset inventory should be accurate, up to date, consistent and aligned with other inventories.

Regarding guidance for PII protection, ISO/IEC 29151 suggests that this invetory of assets should be established and maintained following the information reported in the PIA report, as it is later analysed in \textbf{ISO/IEC 29134}.
Starting from the PIA report, the organisation should extract some relevant informations, such as the types of PII processed, the level of impact of any breach of PII, the purposes of collecting the PII and the retention period of PII.

\vspace{\baselineskip}
With respect to \textbf{Information Classification}, it is suggested to classify general information according to legal requirements and criticality and sensitivity to unauthorised disclosure or modification.
The classification scheme should be consistent across the whole organization so that everyone will classify information 
and related assets in the same way, have a common understanding of protection requirements and apply the appropriate protection.

\vspace{\baselineskip}
For PII protection, ISO/IEC 29151 shows that the classification scheme should consider also all information containing PII.
For instance, may also include more specific categories, such as personal health information or personal financial information.

Some PII that may be classified non-sensitive in one country may be treated as sensitive elsewhere, depending on the 
applicable data protection laws, so as a consequence, the  classification for an element of PII could need re-evaluation and modification when associated with one or more additional attributes.

%section
\subsection{Access Control - User Access Management}
\subsubsection{Objective and Implementation guidance for the protection of PII}
