%section
\subsection{ISO/IEC 29151: Overview of the standard and Purpose}
ISO/IEC 29151 is a globally recognised standard that provides guidelines for protecting personally identifiable information (PII) in information technology.
The primary focus of ISO 29151 is to establish controls and guidelines that organisations can implement to manage privacy risks related to the processing of PII.

\vspace{\baselineskip}
ISO/IEC 29151 is a crucial component of data privacy as it provides a comprehensive framework for the protection of personal data. It is particularly relevant today, as the collection, processing, and storage of personal data have become commonplace. This standard is designed to help organisations manage the privacy of personal data in a systematic and consistent manner, thereby reducing the risk of data breaches and ensuring compliance with data protection regulations.

\vspace{\baselineskip}
The standard applies to all types and sizes of organisations, including public and private companies, government entities, and not-for-profit organisations. It is also applicable to all sectors and industries that process PII, such as healthcare, finance, education, and retail. 
In general, it is applicable to all types and sizes of organizations acting as PII controllers, as defined in ISO/IEC 29100.
The standard is designed to be flexible and can be tailored to each organisation's specific needs and circumstances.



\vspace{\baselineskip}
While implementing ISO/IEC 29151 can provide several benefits, it can also pose some challenges. One of the main challenges is the complexity of the standard, which requires a deep understanding of privacy principles and controls. Implementing the standard can also require significant time and resources, particularly for organisations that do not have a pre-existing privacy management system.

Another challenge in implementing ISO 29151 is the need for cultural change. Protecting PII requires the involvement and commitment of all employees, which can be difficult to achieve in organisations with a low awareness of privacy issues. Furthermore, the standard requires ongoing monitoring and review, which can be challenging to maintain over time. \cite{privacyengine_iso_29151_2025}

It was published on September 28, 2017 and reviewed in 2022.

\begin{figure}[H] 
    \centering
    \includegraphics[width=0.5\textwidth]{images/iso_29151.png} % percorso e scala
    \caption{ISO/IEC 29151}
    \label{fig:iso29151}
\end{figure}

%section
\subsection{Requirments for the Protection of PII}
An organization should identify its PII protection requirements and the privacy principles in ISO/IEC 29100 apply to the identification of requirements.

There are three main sources of PII protection requirements:
\begin{itemize}
    \item legal, statutory, regulatory and contractual requirements related to protection of PII;
    \item assessment of risks (i.e., security risks and privacy risks) to the organization and the PII principal, taking into account the organization’s overall business strategy and objectives, through a risk assessment; 
    \item corporate policies: an organization may also choose voluntarily to go beyond the criteria that are derived from previous requirements.
\end{itemize}

PII protection controls should be selected on the basis of a risk assessment. The results of a 
privacy impact assessment (PIA) (later analysed in ISO/IEC 29134) will help to guide and determine the appropriate 
treatment action and priorities for managing risks to the protection of PII and for implementing controls selected to protect 
against these risks.

%section
\subsection{Structure of the Standard}
It is structured into several sections, each of which covers a specific aspect of privacy controls. The main sections of the standard include the scope, normative references, terms and definitions, privacy principles, and controls.
The standard also includes several annexes that provide additional guidance and examples of privacy controls. These annexes cover topics such as the relationship between ISO 29151 and other standards, the implementation of privacy controls, and the management of privacy risks.


\vspace{\baselineskip}
The controls presented in this standard can be used as reference for organizations that process PII, and the selection of controls  is dependent upon organizational decisions based on the criteria for risk treatment options. 
If required, controls can also be selected from other control sets or new controls can be designed to meet specific needs, as appropriate. 

\vspace{\baselineskip}
The following sections of this report provide a detailed analysis of privacy controls and guidelines defined in ISO/IEC 29151, by following ans adding more controls with respect to the ones defind in ISO/IEC 27002.
The format for this part uses the relevant clause headings and numbering from ISO/IEC 27002 to allow cross-reference to that International Standard. 
For these reason, in this report, some references to ISO/IEC 27002 will be present as well.


%section
\subsection{Information Security Policies}
